\documentclass[10pt,a4paper]{article}
% ========================== DO NOT MODIFY =================================== %
\usepackage[a4paper,left=2.0cm,top=2.0cm,right=2.0cm,bottom=2.0cm]{geometry} 
\usepackage{amssymb,amsmath} % If amsmath is required
\usepackage{url}
\usepackage{times}
\usepackage{mathptmx}      %% use fitting times fonts also in formulas
\usepackage{graphicx}
\usepackage{titling}
\usepackage{authblk}
\usepackage{multicol}
\usepackage{float}
\usepackage{enumitem}
\usepackage[small,compact]{titlesec}
\usepackage[numbers,sort&compress]{natbib}
\usepackage[T1]{fontenc}
\usepackage[UKenglish]{babel}
\usepackage[small,bf,tableposition=top,figureposition=bottom,skip=2pt]{caption}


% limit space between floats and text
\setlength{\textfloatsep}{10pt plus 1.0pt minus 2.0pt}
% shrink the list environments
\setlist{noitemsep}
\setlist{nolistsep}

% Shrink the bibliography
  \let\oldthebibliography=\thebibliography
  \let\endoldthebibliography=\endthebibliography
  \renewenvironment{thebibliography}[1]{%
    \begin{oldthebibliography}{#1}%
      \setlength{\parskip}{0ex}%
      \setlength{\itemsep}{0ex}%
  }%
  {%
    \end{oldthebibliography}%
  }

% Format of the front matter
\renewcommand{\Affilfont}{\small}
\setlength{\affilsep}{1ex}
\renewcommand\Authands{, }
\setlength{\droptitle}{-1.634cm}
% ============================================================================ %


% To be able to type european characters without escape sequences
% uncomment according to your operating system, if desired:
% ---------------------------------------------------------------------------- %
% \usepackage[latin1]{inputenc}   %% (Windows, old Linux)
% \usepackage[utf8]{inputenc}     %% (Linux)
% \usepackage[applemac]{inputenc} %% (Mac OS)
% ---------------------------------------------------------------------------- %


%------------------------------- FRONT MATTER -------------------------------- %
\title{Template for EIT2014%
\vspace{-2ex}} %remove vertical space
\author[1]{Bart\l{}omiej~Grychtol}
\author[2]{Andy~Adler}
\affil[1]{German Cancer Research Center, Heidelberg, Germany}
\affil[2]{Carleton University, Ottawa, Canada, \protect\url{info@eit2014.org}}
\date{}
%----------------------------------------------------------------------------- %



\begin{document}
\maketitle
\vspace{-1.5cm}
\thispagestyle{empty}

\begin{multicols}{2}

\noindent {\bf Abstract:} The abstract should contain no more than 8 
lines of text and describe the main methods and findings.
The abstract should contain no more than 8 
lines of text and describe the main methods and findings.
The abstract should contain no more than 8 
lines of text and describe the main methods and findings.
The abstract should contain no more than 8 
lines of text and describe the main methods and findings.

\section{Introduction}
This is a \LaTeX{} template file to be used for manuscript submissions for
the 14\textsuperscript{th} International Conference on Electrical Impedance 
Tomography
to be held in Gananoque, Canada on April 24-26, 2014. Please note the general 
requirements:
\begin{itemize}
\item Single page of A4 paper
\item All margins at 2~cm
\item British English
\item Main body should be in 10pt Times font
\item Email address of at least one author must be specified as part of the 
affiliation block
\end{itemize}
{\bf The program committee requires a full 1 page submission (rather than
just a short abstract).}


\section{Methods} 
The paper can be prepared in \LaTeX\ or MS Word and submitted as a *.pdf or 
*.docx file, 
respectively. The templates can be downloaded here:%
\begin{itemize}
\item \LaTeX:\\\url{http:\\eit2014.org\tmplt_latex.zip}
\item Word:\\\ \url{http:\\eit2014.org\tmplt_word.zip}
\end{itemize}

\subsection{Figures and tables} 
Figures and tables (floats) can span one or two columns. Note that
\begin{enumerate}
\item Two-column floats must be at the bottom of the page.
\item Captions appear below figures but above tables, as in
fig.~\ref{fig:single-col} and table~\ref{tbl:twocol}.
\end{enumerate}

\begin{figure}[H]
\centering
%\includegraphics[width=.96\columnwidth]{filename.pdf}
\fbox{ \parbox[r][4cm][c]{.96\columnwidth}{ filename.pdf }}
\caption{\label{fig:single-col}%
Column-width figure, using the {\tt figure} environment with the {\tt [H]} 
option from the {\tt floats} package.}
\end{figure}

\subsection{Equations}
Equations should be placed on separate lines and numbered
\begin{equation}
x(t) = s(f_\omega(t))
\label{eq1}
\end{equation}
According to equation \ref{eq1}, a residue theorem states that
\begin{equation}
\oint_C F(z)dz=2 \pi j \sum_k Res[F(z),p_k],
\label{eq3}
\end{equation}
which is an important finding.


\subsubsection{References}
List and number all references at the end of the paper. The references should be 
numbered in order of appearance in the document.
The reference formats for a journal article~\cite{Waspaa1}, a book
~\cite{Waspaa2} and conference proceedings~\cite{other} are illustrated in the 
References section. 

The format is compact and does not include titles for 
articles. References in text should be sorted and grouped compactly, as 
in~\cite{Waspaa1,Waspaa2,other}. We encourage the use of \textsc{Bib}\TeX\ and 
the provided {\tt compact.bst} file as well as the {\tt natbib} package.

\section{Conclusions}
This paper makes many important points.


\footnotesize
\bibliographystyle{compact}
\bibliography{eit2014}
% We encourage the use of bibtex, but you can also type the reference 
% the traditional way
%\begin{thebibliography}{}
%\bibitem[1]{Waspaa1} Lyon RF, Mead, C, IEEE 
%Trans. ASSP  36: 1119--1134, 1988.
%\bibitem[2]{Waspaa2} Lee, KF, Automatic Speech Recognition: The Development 
%of the SPHINX SYSTEM, Kluwer Academic Publishers, Boston, 1989.
%\end{thebibliography}
\end{multicols}



\begin{table*}[b]
\centering
%From: http://amath.colorado.edu/documentation/LaTeX/reference/tables/ex1.html
\caption{\label{tbl:twocol} %
Two-column table, using the {\tt table*} environment placed at the end of the 
document.}
\begin{tabular*}{\textwidth}{@{\extracolsep{\fill}}@{}|c|ccc|r|}
        \hline
$k$ &  $x_1^k$    &   $x_2^k$  & $x_3^k$   & remarks  \\
        \hline
0   & -0.3 & 0.6 & 0.7  &  \\
7   & 0.5 & 0          & -0.523  & $\epsilon < \xi $ \\
        \hline
\end{tabular*}
\end{table*}


\end{document}
